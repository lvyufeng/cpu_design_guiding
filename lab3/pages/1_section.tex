\section{实验三 \quad 简易单周期CPU实验}
MIPS架构CPU的传统流程可分为\textit{取指、译码、执行、访存、回写}(Instruction Fetch, Decode, Execution, Memory Request, Write Back),五阶段。

实验一完成了ALU设计并掌握了存储器IP的使用;实验二实现了单周期CPU的取指、译码阶段,完成了PC、控制器的设计。在实验一与实验二的基础上,单周期CPU的设计的各模块已经具备,再引入数字逻辑课程中所实现的多路选择器、加法器等门级组件,通过对原理图的理解,分析单条(单类型)指令在数据通路中的执行路径,依次连接对应端口,即可完成单周期CPU。

在进行本次实验前,你需要具备以下基础能力:
\begin{enumerate}
    \item 熟悉Vivado的仿真功能(行为仿真)
    \item 理解数据通路、控制器的信号
\end{enumerate}

\subsection{实验目的}
\begin{enumerate}
    \item 掌握不同类型指令在数据通路中的执行路径。
    \item 掌握Vivado仿真方式。
\end{enumerate}
\subsection{实验设备}
\begin{enumerate}
    \item 计算机1台(尽可能达到8G及以上内存);
    \item Xilinx Vivado开发套件(2019.1版本)。
\end{enumerate}
\textcolor{red}{注:本次实验为CPU软核实验,不涉及开发板外围设备,故不需要开发板
}
\subsection{实验任务}
\subsubsection{实验要求}
阅读实验原理实现以下模块:
\begin{enumerate}[(1)]
    \item Datapath,其中主要包含alu(实验一已完成),PC(实验二已完成),adder、mux2、signext、sl2(其中adder、mux2数字逻辑课程已实现,signext、sl2参见实验原理),
    \item Controller(实验二已完成),其中包含两部分,分别为main\_decoder,alu\_decoder。
    \item 指令存储器inst\_mem(Single Port Ram),数据存储器data\_mem(Single Port Ram);使用Block Memory Generator IP构造指令,注意考虑PC地址位数统一。(参考实验二)
    \item 参照实验原理,将上述模块依指令执行顺序连接。实验给出top文件,需兼容top文件端口设定。
    \item 实验给出仿真程序,最终以仿真输出结果判断是否成功实现要求指令。
\end{enumerate}

\subsubsection{实验步骤}
\begin{enumerate}
    \item 从实验一中,导入alu模块;
    \item 从实验二中导入PC、Controller模块;
    \item 从数字逻辑实验中导入多路选择器、加法器模块;
    \item 使用Block Memory,其中inst\_mem导入coe文件;
    \item 参考实验原理,连接各模块;
    \item 导入顶层文件及仿真文件,运行仿真; 
\end{enumerate}

\subsection{实验环境}
\begin{table}[htbp]
    \centering
    \begin{tabular}{|l|l|}
        \hline
        |--top.v& 设计顶层文件,已提供。 \\
        |----mips.v & MIPS软核顶层文件,将Controller与Datapath连接,已提供 \\
        % \textcolor{red}{|----pc.v} &\textcolor{red}{D触发器结构。输入为下一条指令地址,输出为当前指令地址。}  \\
        % |----ram.ip&RAM IP,通过Block memory generator进行实例化。  \\
        \textcolor{blue}{|------controller.v} &\textcolor{blue}{ 控制器模块,本次实验重点。} \\
        \textcolor{blue}{|--------maindec.v} & \textcolor{blue}{Main decoder模块,负责译码得到各个组件的控制信号。} \\
        \textcolor{blue}{|--------aludec.v} & \textcolor{blue}{ALU Decoder模块,负责译码得到ALU控制信号。}\\
        \textcolor{red}{|------datapath.v}& \textcolor{red}{数据通路模块,自行实现} \\
        \textcolor{blue}{|--------pc.v} & \textcolor{blue}{PC模块,使用实验二代码} \\
        \textcolor{blue}{|--------alu.v} & \textcolor{blue}{ALU模块,使用实验一代码} \\
        \textcolor{green}{|--------sl2.v} & \textcolor{green}{移位模块,参考《其他组件实现.pdf》} \\
        \textcolor{green}{|--------signext.v} & \textcolor{green}{有符号扩展模块,参考《其他组件实现.pdf》} \\
        \textcolor{red}{|--------mux2.v} & \textcolor{red}{二选一选择器,自行实现} \\
        \textcolor{black}{|--------regfile.v} & \textcolor{black}{寄存器堆,已提供} \\
        \textcolor{black}{|--------adder.v} & \textcolor{black}{加法器,已提供} \\
        \textit{|----inst\_ram.ip} & \textit{RAM IP,通过Block memory generator进行实例化}\\
        \textit{|----data\_ram.ip} & \textit{RAM IP,通过Block memory generator进行实例化}\\
        \hline
    \end{tabular}
    \caption{实验文件树}
    \label{tab:file_tree}
\end{table}