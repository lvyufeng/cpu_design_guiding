\section{实验四 \quad 流水线MIPS处理器设计}
在实验三中已介绍过MIPS传统的五级流水线阶段,分别是\textit{取指、译码、执行、访存、回写(Instruction Fetch, Decode, Execution, Memory Request, Write Back)},五阶段。

单周期CPU虽然CPI为1,但由于时钟周期取决于时间最长的指令(如lw、sw)没有很好的性能,而多周期虽然能提升性能但仍旧无法满足当今处理器的需求。流水线能够很好地解决效率问题,通过分阶段,达到指令的并行执行。同时,在单周期的基础上,能够很容易地使用触发器做阶段分隔,实现流水线。

本次实验将从实验三单周期处理器过渡至五级流水,并将解决冒险(hazard)问题。

\subsection{实验目的}
\begin{enumerate}
	\item 掌握流水线(Pipelined)处理器的思想;
	\item 掌握单周期处理中执行阶段的划分;
	\item 了解流水线处理器遇到的冒险;
	\item 掌握数据前推、流水线暂停等冒险解决方式。
\end{enumerate}
\subsection{实验设备}
\begin{enumerate}
    \item 计算机1台(尽可能达到8G及以上内存);
    \item Xilinx Vivado开发套件(2019.1版本)。
\end{enumerate}
\textcolor{red}{注:本次实验为CPU软核实验,不涉及开发板外围设备,故不需要开发板
}
\subsection{实验任务}
\subsubsection{实验要求}
阅读实验原理实现以下模块:
\begin{enumerate}[(1)]
    \item Datapath,所有模块均可由实验三复用,需根据不同阶段,修改mux2为mux3(三选一选择器),以及带有enable(使能)、clear(清除流水线)等信号的触发器,
    \item Controller,其中main decoder与alu decoder可直接复用,另需增加触发器在不同阶段进行信号传递
    \item 指令存储器inst\_mem(Single Port Ram),数据存储器data\_mem(Single Port Ram);同实验三一致,无需改动,
    \item 参照实验原理,在单周期基础上加入每个阶段所需要的触发器,重新连接部分信号。实验给出top文件,需兼容top文件端口设定。
    \item 实验给出仿真程序,最终以仿真输出结果判断是否成功实现要求指令。
\end{enumerate}

\subsubsection{实验步骤}
\begin{enumerate}
    \item 在实验三基础上,加入D触发器,增加流水级数;
    \item 处理流水线冒险问题,加入暂停和前推机制;
    \item 导入顶层文件及仿真文件,运行仿真; 
\end{enumerate}

\subsection{实验环境}
\begin{table}[htbp]
    \centering
    \begin{tabu}{|l|l|}
        \hline
        |--top.v& 设计顶层文件,已提供。 \\
        |----mips.v & MIPS软核顶层文件,将Controller与Datapath连接,已提供 \\
        % \textcolor{red}{|----pc.v} &\textcolor{red}{D触发器结构。输入为下一条指令地址,输出为当前指令地址。}  \\
        % |----ram.ip&RAM IP,通过Block memory generator进行实例化。  \\
        \textcolor{blue}{|------controller.v} &\textcolor{blue}{ 控制器模块,本次实验重点。} \\
        \textcolor{blue}{|--------maindec.v} & \textcolor{blue}{Main decoder模块,负责译码得到各个组件的控制信号。} \\
        \textcolor{blue}{|--------aludec.v} & \textcolor{blue}{ALU Decoder模块,负责译码得到ALU控制信号。}\\
        \textcolor{red}{|------datapath.v}& \textcolor{red}{数据通路模块,自行实现} \\
        \textcolor{blue}{|--------pc.v} & \textcolor{blue}{PC模块,使用实验二代码} \\
        \textcolor{blue}{|--------alu.v} & \textcolor{blue}{ALU模块,使用实验一代码} \\
        \textcolor{green}{|--------sl2.v} & \textcolor{green}{移位模块,参考《其他组件实现.pdf》} \\
        \textcolor{green}{|--------signext.v} & \textcolor{green}{有符号扩展模块,参考《其他组件实现.pdf》} \\
        \textcolor{red}{|--------mux2.v} & \textcolor{red}{二选一选择器,自行实现} \\
        \textcolor{black}{|--------regfile.v} & \textcolor{black}{寄存器堆,已提供} \\
        \textcolor{black}{|--------adder.v} & \textcolor{black}{加法器,已提供} \\
        \rowfont{\color{black}}
        |--------floprc & 带有reset与clear的触发器 \\
        \rowfont{\color{black}}
	    |--------flopenr & 带有enable与reset的触发器 \\
	    \rowfont{\color{black}}
	    |--------flopenrc & 带有enable、reset与clear的触发器 \\
	    \rowfont{\color{red}}
        |--------hazard & 冒险处理模块,自行实现 \\
        \textit{|----inst\_ram.ip} & \textit{RAM IP,通过Block memory generator进行实例化}\\
        \textit{|----data\_ram.ip} & \textit{RAM IP,通过Block memory generator进行实例化}\\
        \hline
    \end{tabu}
    \caption{实验文件树}
    \label{tab:file_tree}
\end{table}