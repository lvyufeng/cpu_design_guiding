\section{Verilog不同实现方式}
验证实验给出两种不同的组合逻辑实现方式:

\begin{lstlisting}[language=Verilog,label=lst:assign,caption=assign实现组合逻辑,numbers=left,xleftmargin=5em,xrightmargin=5em, aboveskip=2em]
module calculate(
    input wire [7:0] num1,
    input wire [2:0] op,
    output [31:0] result
    );
    wire [31:0] num2;
    wire [31:0] Sign_extend;
    
    assign num2 = 32'h00000001;
    assign Sign_extend={{24{1'b0}},num1[7:0]};
    assign result = (op == 3'b000)? Sign_extend + num2:
                    (op == 3'b001)? Sign_extend - num2:
                    (op == 3'b010)? Sign_extend & num2:
                    (op == 3'b011)? Sign_extend | num2:
                    (op == 3'b100)? ~Sign_extend: 32'h00000000;
endmodule
\end{lstlisting}


\begin{lstlisting}[language = Verilog,label=lst:always,caption=always实现组合逻辑,numbers=left,xleftmargin=5em,xrightmargin=5em, aboveskip=2em]
module calculate(
    input wire [7:0] num1,
    input wire [2:0] op,
    output reg [31:0] result
    );
    reg [31:0] num2;
    reg[31:0] Sign_extend;
    always @(op) begin
       num2 = 32'h00000001;
       Sign_extend={24'h000000,num1[7:0]};
            case(op)
                3'b000:result = Sign_extend + num2;
                3'b001:result = Sign_extend - num2; 
                3'b010:result = Sign_extend &  num2;
                3'b011:result = Sign_extend |  num2;
                3'b100:result = ~Sign_extend  ;          
                default:result = 32'h00000000;
            endcase
        end
endmodule
\end{lstlisting}

二者区别对比如下:

\begin{table}[htbp]
    \centering
    \begin{tabular}{c|c|c}
          \hline
          & assign & always  \\
          \hline
          result    & wire & reg \\
          num2      & wire  & reg \\
          sign\_extend & wire &  reg\\
          \hline
    \end{tabular}
    \caption{assign \& always 信号对比}
    \label{tab:sig_compare}
\end{table}
Always即可实现组合逻辑,也可实现时序逻辑:
\begin{enumerate}[(1)]
    \item always @(a or b or c)或always@(*)形式的,即不带时钟边沿的,综合出来还是组合逻辑;
    \item always @(posedge clk)形式的,即带有边沿的,综合出来一般是时序逻辑,会包含触发器(Flip-Flop)
\end{enumerate}

观察Listing~\ref{lst:always}中always的触发信号op,可以得知其为组合逻辑。在这里,给出两种方式在FPGA资源及编程难度上的对比。


\begin{table}[htbp]
    \centering
    \begin{tabular}{c|p{5cm}|p{5cm}}
         \hline
          & assign & always \\
         \hline
         资源占用  &  使用wire变量,综合得到的组合逻辑全部由导线构成,不占用FPGA资源 & 由于内部赋值与输出都需要使用reg变量,同为组合逻辑,需占用一定资源 \\
         \hline
         编程及理解难度 & 面向信号进行状态描述,一般需要考虑单个信号在所有状态下的可能性,需要完全理解后进行编程。
        
        阅读者理解难度大。
        & 面向实验需求描述,always模块内书写方式与高级语言相仿,可以按照实验要求逐一列举。

        阅读者理解较容易。
        \\
         \hline
    \end{tabular}
    \caption{assign \& always 设计对比}
    \label{tab:design_compare}
\end{table}

\textcolor{red}{\textbf{注意:}}\textbf{Imagination及龙芯高校开源计划代码均采取assign方式,实验推荐使用assign方式实现组合逻辑。设计实验中新增内容使用assign方式增加输出信号更为简单。}