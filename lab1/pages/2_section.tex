\section{流水线}
\subsection{流水线原理}
\subsubsection{什么是流水线}
流水线设计就是将组合逻辑系统地分割,并在各个部分(分级)之间插入寄存器,并暂存中间数据的方法。目的是将一个大操作分解成若干的小操作,每一步小操作的时间较小,所以能提高频率,各小操作能并行执行,所以能提高数据吞吐率(提高处理速度)。

\subsubsection{什么时候用流水线设计}
使用流水线一般是时序比较紧张,对电路工作频率较高的时候。典型情况如下:
\begin{enumerate}
    \item 功能模块之间的流水线,用乒乓 buffer 来交互数据。代价是增加了 memory 的数量,但是和获得的巨大性能提升相比,可以忽略不计。
    \item I/O 瓶颈,比如某个运算需要输入 8 个数据,而 memroy 只能同时提供 2个数据,如果通过适当划分运算步骤,使用流水线反而会减少面积。
    \item 片内 sram的读操作,因为sram的读操作本身就是两极流水线,除非下一步操作依赖读结果,否则使用流水线是自然而然的事情。
    \item 组合逻辑太长,比如(a+b)*c,那么在加法和乘法之间插入寄存器是比较稳妥的做法。
\end{enumerate}

\subsubsection{使用流水线的优缺点}

\textbf{优点}:流水线缩短了在一个时钟周期内给的那个信号必须通过的通路长度,增加了数据吞吐量,从而可以提高时钟频率,但也导致了数据的延时。
% 举例如下:

% 一个 2 级组合逻辑,假定每级延迟相同为 Tpd,
% \begin{enumerate}
%     \item 无流水线的总延迟就是2Tpd,可以在一个时钟周期完成,但是时钟周期受限制在 2Tpd;
%     \item 流水线:
%     每一级加入寄存器(延迟为 Tco)后,单级的延迟为 Tpd+Tco,每级消耗一个时钟周期,流水线需要 2 个时钟周期
%     来获得第一个计算结果,称 为首次延迟,它要 2*( Tpd+Tco),但是执行重复操作时,只要一个时钟周期来获得最后的
%     计算结果,称为吞吐延迟( Tpd+Tco)。可见只要 Tco 小于 Tpd,流水线就可以提高速度。 特别需要说明的是,流水线
%     并不减小单次操作的时间,减小的是整个数据的操作时间,请大家认真体会。
% \end{enumerate}

\textbf{缺点}: 功耗增加,面积增加,硬件复杂度增加,特别对于复杂逻辑如 CPU 的流水线而言,流水越深,发生需要暂停流水线或刷新流水线的情况时,时间损失越大。

所以使用流水线并非有利无害,设计时需权衡考虑。


\subsection{流水线基础示例}
首先要说明的一点是,流水线电路纯粹就是一个数字电路的概念,不要一谈到流水线就仅仅认为是处理器中 的流水线。

我们先来看一个完全不会被阻塞的 3 级流水线电路怎么写。
\begin{lstlisting}[language=Verilog,label=lst:3_stage_pipeline,caption=无阻塞3级流水线实现,numbers=left,xleftmargin=5em,xrightmargin=5em, aboveskip=2em]
module non_stall_pipeline #
(
    parameter WIDTH = 100
)
(
    input              clk,
    input  [WIDTH-1:0] datain,
    output [WIDTH-1:0] dataout
);

reg [WIDTH-1:0] pipe1_data;
reg [WIDTH-1:0] pipe2_data;
reg [WIDTH-1:0] pipe3_data;

always @(posedge clk) begin
    pipe1_data <= datain;
end

always @(posedge clk) begin
    pipe2_data <= pipe1_data;
end

always @(posedge clk) begin
    pipe3_data <= pipe2_data;
end

assign dataout = pipe3_data;

endmodule
\end{lstlisting}

但是实际电路设计中,并不总是上面这种无阻塞的流水线。下面给出一个具有阻塞功能的流水线代码。
\begin{lstlisting}[language=Verilog,label=lst:3_stage_pipeline_stall_flush,caption=有阻塞3级流水线实现,numbers=left,xleftmargin=5em,xrightmargin=5em, aboveskip=2em]
module stallable_pipeline #
(
parameter WIDTH = 100
)
(
input              clk,
input              rst,
input              validin,
input  [WIDTH-1:0] datain,
input              out_allow,
output             validout,
output [WIDTH-1:0] dataout
);

reg             pipe1_valid;
reg [WIDTH-1:0] pipe1_data;
reg             pipe2_valid;
reg [WIDTH-1:0] pipe2_data;
reg             pipe3_valid;
reg [WIDTH-1:0] pipe3_data;

// pipeline stage1
wire            pipe1_allowin;
wire            pipe1_ready_go;
wire            pipe1_to_pipe2_valid;

assign pipe1_ready_go = ……
assign pipe1_allowin = !pipe1_valid || pipe1_ready_go && pipe2_allowin;
assign pipe1_to_pipe2_valid = pipe1_valid && pipe1_ready_go;

always @(posedge clk) begin
    if (rst) begin
        pipe1_valid <= 1’b0;
    end
    else if (pipe1_allowin) begin
        pipe1_valid <= validin;
    end
    if (validin && pipe1_allowin) begin
        pipe1_data <= datain;
    end
end

// pipeline stage2
wire            pipe2_allowin;
wire            pipe2_ready_go;
wire            pipe2_to_pipe3_valid;

assign pipe2_ready_go = ……
assign pipe2_allowin = !pipe2_valid || pipe2_ready_go && pipe3_allowin;
assign pipe2_to_pipe3_valid = pipe2_valid && pipe2_ready_go;

always @(posedge clk) begin
    if (rst) begin
        pipe2_valid <= 1’b0;
    end
    else if (pipe2_allowin) begin
        pipe2_valid <= pipe1_to_pipe2_valid;
    end
    if (pipe1_to_pipe2_valid && pipe2_allowin) begin
        pipe2_data <= pipe1_data;
    end
end

// pipeline stage3
wire            pipe3_allowin;
wire            pipe3_ready_go;

assign pipe3_ready_go = ……
assign pipe3_allowin = !pipe3_valid || pipe3_ready_go && out_allow;

always @(posedge clk) begin
    if (rst) begin
        pipe3_valid <= 1’b0;
    end
    else if (pipe3_allowin) begin
        pipe3_valid <= pipe2_to_pipe3_valid;
    end
    if (pipe2_to_pipe3_valid && pipe3_allowin) begin
        pipe3_data <= pipe2_data;
    end
end

assign validout = pipe3_valid && pipe3_ready_go;
assign dataout  = pipe3_data;

endmodule
\end{lstlisting}

上述代码表述中pipe\textbf{X}\_valid为1表示第X级流水级上当前存有有效的数据,为0表示第X级是空的。因为控制位的好处是,清空流水线的时候不用刷各级流水线data域的值,可以节约逻辑资源,但是需要注意的是根据流水 级的data域信息产生控制信号时,有时候不要忘记“与上”这一级的 valid 信号。

pipe\textbf{X}\_allowin为1表示第X级可以接收前一级流水送来的数据。

上面流水线的 RTL 写法对于流水级的关键控制信号采用了链式写法。这样写的好处是,如果发生流水级功能调整时,譬如将一级再切分为两级,或者将某一级的停顿周期数改变了,都只需要修改局部,不会涉及全局性的 调整。有的同学可能会觉得我这种写法很麻烦,觉得自己写出来的简单一些。请你们针对具体情况分析,是否对于你所涉及的流水线,上面给出的代码中,某些pipeX\_allowin和pipeX\_ready\_go可以恒置为1,将其代入上面的表达式中,并将级联的控制信号完全展开,就形成了你们的那种简单版本。这里给出这样的一个流水线电路的“模板”式写法,主要是因为它易于理解不易出错,当同学们在自己进行流水线设计时如果因为流水级控制信号 搞得很糊涂时,可以回过头来看看此处给出的例子。

再额外说一点,上面的写法是把流水线的 control logic 和 data path 写到了一个模块内部。你也可以把两者分在不同的模块中。国外的教科书上在描述流水线 CPU 时,通常会在流水线之外画一个叫做 pipeline control logic 的椭圆形框,从这个框引出控制信号去往各级流水所在触发器的使能端。这就是把 control logic 和 data path 分离的写法。 我们在上面给出的例子,pipeX\_data 就是 data path主体。
 
\subsection{8bit全加器示例}
\subsubsection{非流水线方式}
\begin{lstlisting}[language=Verilog,label=lst:no_stage,caption=8bit全加器非流水线实现,numbers=left,xleftmargin=5em,xrightmargin=5em, aboveskip=2em]
module adder_8bits(
    a, 
    b, 
    c
);
input  [7:0] a;
input  [7:0] b;
output [8:0] c;

assign c[8:0] = {1'd0, a} + {1'd0, b};
endmodule
\end{lstlisting}
\subsubsection{2级流水线}
\begin{lstlisting}[language=Verilog,label=lst:2_stage_pipeline,caption=8bit全加器2级流水线实现,numbers=left,xleftmargin=5em,xrightmargin=5em, aboveskip=2em]
module adder_4bits_2steps(cin_a, cin_b, cin, clk, cout, sum);
input [7:0]     cin_a;
input [7:0]     cin_b;
input           cin;
input           clk;
output          cout;
output [7:0]    sum;
	 
reg         cout;
reg         cout_temp;
reg [7:0]   sum;
reg [3:0]   sum_temp;
	 
always @(posedge clk) begin
    {cout_temp,sum_temp} <= cin_a[3:0] + cin_b[3:0] + cin;
end
	 
always @(posedge clk) begin
    {cout,sum} <= {{1'b0,cin_a[7:4]} + {1'b0,cin_b[7:4]} + cout_temp, sum_temp};
end
endmodule
\end{lstlisting}
\subsubsection{4级流水线}
\begin{lstlisting}[language=Verilog,label=lst:4_stage_pipeline,caption=8bit全加器4级流水线实现,numbers=left,xleftmargin=5em,xrightmargin=5em, aboveskip=2em]
module adder_8bits_4steps(cin_a, cin_b, c_in, clk, c_out, sum_out);
input [7:0]     cin_a;
input [7:0]     cin_b;
input           c_in;
input           clk;
output          c_out;
output [7:0]    sum_out;
	 
reg c_out;
reg c_out_t1, c_out_t2, c_out_t3;
	 
reg [7:0] sum_out;
reg [1:0] sum_out_t1;
reg [3:0] sum_out_t2;
reg [5:0] sum_out_t3;
	 
always @(posedge clk) begin
    {c_out_t1, sum_out_t1} <= {1'b0, cin_a[1:0]} + {1'b0, cin_b[1:0]} + c_in;
end
	 
always @(posedge clk) begin
    {c_out_t2, sum_out_t2} <= {{1'b0, cin_a[3:2]} + {1'b0, cin_b[3:2]} + c_out_t1, sum_out_t1};
end
	 
always @(posedge clk) begin
    {c_out_t3, sum_out_t3} <= {{1'b0, cin_a[5:4]} + {1'b0, cin_b[5:4]} + c_out_t2, sum_out_t2};
end
	 
always @(posedge clk) begin
    {c_out, sum_out} <= {{1'b0, cin_a[7:6]} + {1'b0, cin_b[7:6]} + c_out_t3, sum_out_t3};
end

endmodule
\end{lstlisting}
