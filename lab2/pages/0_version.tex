\textbf{版本}

\begin{table}[htbp]
    \centering
    \begin{tabular}{|c|c|c|c|p{6cm}|}
        \hline
        \multirow{4}{*}{\textbf{更新记录}} & \multicolumn{2}{c|}{文档名} & \multicolumn{2}{c|}{实验指导书\_lab2} \\
        % \hline
        \cline{2-5} & \multicolumn{2}{c|}{版本号} & \multicolumn{2}{c|}{0.3} \\
        \cline{2-5} & \multicolumn{2}{c|}{创建人} & \multicolumn{2}{c|}{计算机组成原理教学组} \\
        \cline{2-5} & \multicolumn{2}{c|}{创建日期} & \multicolumn{2}{c|}{2017/10/18} \\
        \hline
        \multicolumn{5}{|l|}{\textbf{更新历史}} \\
        \hline
        \textbf{序号} & \textbf{更新日期} & \textbf{更新人} & \textbf{版本号} & \textbf{更新内容} \\
        \hline 
        1 & 2017/10/8 & 吕昱峰 & 0.1 & 初版,单周期CPU取指译码实验\\
        \hline
        2 & 2018/10/8 & 吕昱峰 & 0.2 & 重新补充译码的理解细节。\\
        \hline
        3 &2019/11/14 & 吕昱峰 & 0.3 & 调整存储器部分至实验2\\
        \hline
         & & & & \\
        \hline
    \end{tabular}
    % \caption{Caption}
    \label{tab:guide_book_version}
\end{table}



% \textbf{致谢:}

% 本文档流水线设计部分采用中国科学院大学实验指导书中流水线基础内容,其设计与Demo均取自龙芯《计算机体系结构导教班》提供的资源,特此说明并感谢。

\textbf{文档错误反馈:}
本文档经多版本迭代,但由于作者精力能力有限,其中出现错误请联系:

\url{lvyufeng@cqu.edu.cn}
